% Template for PLoS
% Version 3.6 Aug 2022
%
% % % % % % % % % % % % % % % % % % % % % %
%
% This template contains comments intended 
% to minimize problems and delays during our production 
% process. Please follow the template instructions
% whenever possible.
%
% Please contact latex@plos.org with any questions.
%
% % % % % % % % % % % % % % % % % % % % % % %

\documentclass[10pt,letterpaper]{article}
\usepackage[top=0.85in,left=2.75in,footskip=0.75in]{geometry}

% amsmath and amssymb packages, useful for mathematical formulas and symbols
\usepackage{amsmath,amssymb}

% Use adjustwidth environment to exceed column width (see example table in text)
\usepackage{changepage}

% textcomp package and marvosym package for additional characters
\usepackage{textcomp,marvosym}

% cite package, to clean up citations in the main text. Do not remove.
\usepackage{cite}

% Use nameref to cite supporting information files (see Supporting Information section for more info)
\usepackage{nameref,hyperref}

% line numbers
\usepackage[right]{lineno}

% ligatures disabled
\usepackage[nopatch=eqnum]{microtype}
\DisableLigatures[f]{encoding = *, family = * }

% color can be used to apply background shading to table cells only
\usepackage[table]{xcolor}

% array package and thick rules for tables
\usepackage{array}

% create "+" rule type for thick vertical lines
\newcolumntype{+}{!{\vrule width 2pt}}

% create \thickcline for thick horizontal lines of variable length
\newlength\savedwidth
\newcommand\thickcline[1]{%
  \noalign{\global\savedwidth\arrayrulewidth\global\arrayrulewidth 2pt}%
  \cline{#1}%
  \noalign{\vskip\arrayrulewidth}%
  \noalign{\global\arrayrulewidth\savedwidth}%
}

% \thickhline command for thick horizontal lines that span the table
\newcommand\thickhline{\noalign{\global\savedwidth\arrayrulewidth\global\arrayrulewidth 2pt}%
\hline
\noalign{\global\arrayrulewidth\savedwidth}}


% Remove comment for double spacing
\usepackage{setspace} 
\doublespacing

% Text layout
\raggedright
\setlength{\parindent}{0.5cm}
\textwidth 5.25in 
\textheight 8.75in

% Bold the 'Figure #' in the caption and separate it from the title/caption with a period
% Captions will be left justified
\usepackage[aboveskip=1pt,labelfont=bf,labelsep=period,justification=raggedright,singlelinecheck=off]{caption}
\renewcommand{\figurename}{Fig}

% Use the PLoS provided BiBTeX style
\bibliographystyle{plos2015}

% Remove brackets from numbering in List of References
\makeatletter
\renewcommand{\@biblabel}[1]{\quad#1.}
\makeatother

% Header and Footer with logo
\usepackage{lastpage,fancyhdr,graphicx}
\usepackage{epstopdf}
%\pagestyle{myheadings}
\pagestyle{fancy}
\fancyhf{}
%\setlength{\headheight}{27.023pt}
%\lhead{\includegraphics[width=2.0in]{PLOS-submission.eps}}
\rfoot{\thepage/\pageref{LastPage}}
\renewcommand{\headrulewidth}{0pt}
\renewcommand{\footrule}{\hrule height 2pt \vspace{2mm}}
\fancyheadoffset[L]{2.25in}
\fancyfootoffset[L]{2.25in}
\lfoot{\today}

%% Include all macros below

\newcommand{\lorem}{{\bf LOREM}}
\newcommand{\ipsum}{{\bf IPSUM}}

%% END MACROS SECTION


\begin{document}
\vspace*{0.2in}

% Title must be 250 characters or less.
\begin{flushleft}
{\Large
\textbf\newline{Ten simple rules for organising an effective (student-led) writing retreat}
}
\newline
% Insert author names, affiliations and corresponding author email (do not include titles, positions, or degrees).
\\
Nicholas W. Daudt\textsuperscript{1,2*},
Claudia Hird\textsuperscript{1,3},
Eleanor Kelly\textsuperscript{1},
Elli Leinikki\textsuperscript{1},
Gretchen McCarthy\textsuperscript{1},
Ian S. Dixon-Anderson\textsuperscript{1},
Jackson Beagley\textsuperscript{1,4},
Jessica Moffitt\textsuperscript{1},
Joseph Curtis\textsuperscript{1},
Lindsay Wickman\textsuperscript{1\textcurrency},
Meghan Duffy\textsuperscript{1,4},
Preston Maluafiti\textsuperscript{1},
Saskia Foreman\textsuperscript{1},
William Carome\textsuperscript{1},
Leah M. Crowe\textsuperscript{1,2}
\\
\bigskip
\textbf{1} Department of Marine Science, University of Otago, Dunedin, Aotearoa/New Zealand
\\
\textbf{2} Department of Mathematics and Statistics, University of Otago, Dunedin, Aotearoa/New Zealand
\\
\textbf{3} Department of Zoology, University of Otago, Dunedin, Aotearoa/New Zealand
\\
\textbf{4} Department of Geology, University of Otago, Dunedin, Aotearoa/New Zealand
\\
\bigskip

% Insert additional author notes using the symbols described below. Insert symbol callouts after author names as necessary.
% 
% Remove or comment out the author notes below if they aren't used.
%
% Primary Equal Contribution Note
% \Yinyang These authors contributed equally to this work.

% Additional Equal Contribution Note
% Also use this double-dagger symbol for special authorship notes, such as senior authorship.
% \ddag These authors also contributed equally to this work.

% Current address notes
\textcurrency Current Address: Geospatial Ecology of Marine Megafauna Laboratory, Marine Mammal Institute, Oregon State University, Newport, OR, USA \& Department of Fisheries, Wildlife, and Conservation Sciences, Oregon State University, Corvallis, OR, USA

% Use the asterisk to denote corresponding authorship and provide email address in note below.
* nicholaswdaudt@gmail.com 

\end{flushleft}

% Please keep the abstract below 300 words
%\section*{Abstract}
%Lorem ipsum dolor sit amet, consectetur adipiscing elit. Curabitur eget porta erat. Morbi consectetur est vel gravida pretium. Suspendisse ut dui eu ante cursus gravida non sed sem. Nullam sapien tellus, commodo id velit id, eleifend volutpat quam. Phasellus mauris velit, dapibus finibus elementum vel, pulvinar non tellus. Nunc pellentesque pretium diam, quis maximus dolor faucibus id. Nunc convallis sodales ante, ut ullamcorper est egestas vitae. Nam sit amet enim ultrices, ultrices elit pulvinar, volutpat risus.


% Please keep the Author Summary between 150 and 200 words
% Use first person. PLOS ONE authors please skip this step. 
% Author Summary not valid for PLOS ONE submissions.   
%\section*{Author summary}
%Lorem ipsum dolor sit amet, consectetur adipiscing elit. Curabitur eget porta erat. Morbi consectetur est vel gravida pretium. Suspendisse ut dui eu ante cursus gravida non sed sem. Nullam sapien tellus, commodo id velit id, eleifend volutpat quam. Phasellus mauris velit, dapibus finibus elementum vel, pulvinar non tellus. Nunc pellentesque pretium diam, quis maximus dolor faucibus id. Nunc convallis sodales ante, ut ullamcorper est egestas vitae. Nam sit amet enim ultrices, ultrices elit pulvinar, volutpat risus.

\linenumbers

\section*{Introduction}

At every stage in a researcher's career, scholarly output advances scientific knowledge and supports career development. Early career researchers, in particular, significantly boost their career prospects by increasing their scholarly outputs ~\cite{horta2016, wilkins2021}. Writing serves as an integral skill for academic work ~\cite{boice1984, hazelett2025}, especially when competing for grants and jobs. Academics juggle administrative tasks alongside teaching, collection and analysis of data, and production of publications and presentations. Consequently, many report a lack of time to think critically as a major challenge in academia ~\cite{menzies2007}, which too often leads researchers to deprioritise writing tasks ~\cite{boice1984, menzies2007}. Therefore, to fully engage in the act of writing, many need to fully disengage from other tasks by carving out dedicated focus time ~\cite{murray2009, murray2013}.

Writing retreats provide structured periods where researchers dedicate time to focused writing ~\cite{mcgrail2006, murray2009}. These retreats offer practical opportunities to disconnect from daily work routines ~\cite{murray2013, tremblay2021}, which help researchers gain writing momentum and increase scholarly output ~\cite{mcgrail2006, kornhaber2016}. In addition, writing retreats foster a sense of community, promote wellbeing, and build self-confidence for academic writers ~\cite{kornhaber2016, eardley2021, tremblay2021}. Postgraduate students, in particular, highlight the value of these retreats in strengthening bonds among peers, obtaining and providing constructive feedback, and dedicating time and space to focus on writing ~\cite{kornhaber2016, papen2018, tremblay2021, bojovic2024}.

As a cohort of PhD students in the Department of Marine Science at the University of Otago (Aotearoa/New Zealand), we organised a 5-day writing retreat at a remote field station in 2023. In New Zealand, PhDs follow research-only programs; as such, we do not participate in coursework that might promote group cohesion, as each student leads their own research. In addition, our diverse disciplines, field sites, and lab locations present challenges in maintaining social cohesion within our programme. A writing retreat was planned to not only facilitate community building within our PhD cohort ~\cite{omeara2017, bernery2022}, but to also structure a productive week as we worked toward our dissertation goals ~\cite{mcgrail2006, kornhaber2016}. The retreat was a success both in terms of writing produced as well as connections built between peers sharing a similar PhD journey. The retreat's success motivated us to organise a second retreat the following year; the second retreat's success inspired a third.

Based on our experience, we present Ten Simple Rules for organising effective (student-led) writing retreats. Although the authors were all PhD students at the time they participated in the retreats, these rules can be applied broadly to any research-oriented or academically-minded group. We outline steps to support the planning and execution of pre- (Rules 1--4), during (Rules 5--9), and post- (Rule 10) writing retreat actions, but we do not cover writing techniques per se. Many helpful resources on academic writing exist (e.g., ~\cite{turbek2016, hotaling2020, carter2020, cargill2021}), including articles in this series ~\cite{weinberger2015, mensh2017, peterson2018}.

\section*{Rule 1: Leverage university facilities}

A change of scenery inspires productivity ~\cite{murray2013}, whether you organise the writing retreat locally or as a multi-day trip. When assessing suitability of the location, also consider internet access, heating/cooling, sleeping arrangements, needed accommodations for participants with particular needs, and workspace infrastructure (power points, seats, tables, etc.). Taking advantage of institutional resources simplifies the planning process and enhances the success of your writing retreat. Local facilities, such as meeting rooms and other bookable spaces on campus, offer accessible options. For longer retreats, students---regardless of departmental affiliation---should inquire about their university's field stations. For example, many biology and ecology programmes maintain field stations (for a global list, see ~\cite{tydecks2016}). By using university-owned facilities, students can likely keep retreat costs manageable. 

Field stations often remain underutilized during certain periods, so retreats during these times can be used to promote year-round upkeep of facilities. In addition, field stations often sit in remote, natural settings with limited distractions, creating ideal environments for focus and uninterrupted work ~\cite{murray2013, tremblay2021}. If your university lacks a suitable field station, other institutions may allow shared use depending on availability. We encourage students to contact station managers and build those connections.

For our writing retreat, we made use of our department's marine field station in Oban, Rakiura/Stewart Island (a small island off the southern coast of New Zealand's South Island). This location suited us well---everything we needed was within walking distance. Because this is a small community (around 400 year-round residents) abutting a national forest, we had limited distractions, straightforward options for restaurants and groceries, and ample access to walking tracks and beaches.

\section*{Rule 2: Prepare a proposal to secure support}

Begin planning your retreat by preparing a proposal that outlines the type of event you aim to organise and gauges interests from participants. Clearly define the capacity and target audience to plan details of the retreat effectively. Your options for facilities (see Rule 1) may determine your maximum number of participants. In our case, we hosted all retreats with 10 participants; we found this number worked well for the size of our facility and the length of the retreat.

After outlining potential locations, dates, and number of attendees, calculate the associated costs. These may include transportation, food, accommodation, and amenities such as internet.

Use the proposal to seek support. Academic researchers may access departmental or divisional funding to support these opportunities for development. There may be external funding available to provide supplemental support if internal money is not available to cover the entire costs of the retreat. This funding could be sought from industry and community organisations (e.g., Lions and Rotary Clubs), or through fundraisers. A well-developed proposal lays a solid foundation for organising the retreat and securing the support needed for success.

\section*{Rule 3: Structure your retreat}

Create a schedule for your writing retreat, as it provides the structure for focused writing and thinking time ~\cite{murray2009, tremblay2021}. Use available resources to guide your planning (e.g., ~\cite{tremblay2021}), including this guide (see also Appendices). At our university, we collaborated with the Higher Education Development Centre, which provided guidance on designing productive days and structuring the week for our first and second retreats.

The first and last day of the schedule were dedicated to travelling to and from the retreat location, allowing for the five days in between to remain fully structured for the retreat activities. We built in extra time on travel days to settle in, pack up, shop for groceries, and explore the area. Upon arrival, we held a welcome discussion to establish ground rules to ensure everyone felt comfortable and had a shared understanding of expectations (see Rule 5). Each work day was themed to provide guidance and structure for participants (Table~\ref{table1}), while allowing each individual to pursue their own goals. We also incorporated optional social activities throughout the week (see Rule 9), such as a pub quiz, wildlife tours, game night, and local events.

% Place tables after the first paragraph in which they are cited.
\begin{table}[!ht]
%\begin{adjustwidth}{-2.25in}{0in} % Comment out/remove adjustwidth environment if table fits in text column.
\centering
\caption{{\bf Examples of daily themes over a 5-day writing retreat. Note that days 1 and 7 were organised for travel.}}
\begin{tabular}{p{1in}p{2in}p{2in}}
\hline
{\bf Day} & {\bf Theme} & {\bf Description}\\ \thickhline
Day 2 & Getting started & Commit to the goal for the week (see Rule 4), organize a strategy, and establish accountability partners.\\ \hline
Day 3 & Words on the page & Just start writing something! (Rule 6)\\ \hline
Day 4 & Keep going! & Continue writing and refine. (Rule 6)\\ \hline
Day 5 & Good enough & Exchange work within accountability partners to practice giving and receiving comments. Use this opportunity to reflect on writing structure and style from a reader's perspective. (Rule 8)\\ \hline
Day 6 & Wrap it up & End the retreat by reflecting on what was accomplished this week (Rule 10) and develop a plan on how to proceed.\\ \hline
\end{tabular}
\label{table1}
%\end{adjustwidth}
\end{table}

Our daily schedule typically ran from 9:00 AM to 6:00 PM and was shared in advance so participants knew what to expect during the week ~\cite{tremblay2021} (see also \nameref{S1_Appendix}). A different facilitator was designated each day to lead discussions and manage writing blocks (60--120 minutes) and breaks (30--60 minutes). Each morning began with a 15-minute writing session using prompts that ranged from reflective writing to humorous research titles (see Rule 9). These sessions helped prime our writing engines for the day and offered a fun, low-pressure opportunity to share writing. Each evening, the day was closed with a reflective discussion on how the day went (with an `accountability partner'; see Rule 5 in ~\cite{peterson2018}), a shared dinner, and presentations (see Rule 7).

\section*{Rule 4: Have a pre-retreat meeting}

Host a pre-retreat meeting with all attendees to ensure a smoother and more productive retreat. This meeting gives participants a chance to meet, finalise logistics, set goals, and raise questions or concerns. This time can be used to build or reinforce a respectful and accountable group dynamic (see Rules 5 and 10), and collaboratively fine-tune logistics and schedules. Most importantly, a pre-retreat meeting allows participants to identify or set specific writing goals while there is still time to prepare relevant resources (literature, data analysis, input from academic supervisors). A pre-retreat meeting can help to prompt participants to organise themselves before leaving so that non-writing activities, like gathering references, do not become distractions during the initial days of the retreat.

We suggest that participants arrive at the pre-retreat meeting with a preliminary goal to share with the group. Goals may be quantitative (e.g., word count, page number) or qualitative (e.g., specific sections or revisions). For reference, during our 5-day retreats, participants averaged 1.2 pages or 550 words per day, working on everything from detailed revisions to drafting full manuscripts. By identifying a specific writing target ahead of time, friction related to making decisions is reduced during the retreat with goals clear in mind. Sharing goals also helps participants learn about each other's research and assess whether goals are realistic.

The pre-retreat meeting should be held within two weeks of departure to support timely preparation while preserving participant momentum. If possible, a casual, in-person meeting is preferred to provide a comfortable setting for participants to meet and connect. Additional objectives may include selecting writing prompts, presentation topics, icebreakers, roles (e.g., meal coordination, daily facilitator), or ideas for social activities.

\section*{Rule 5: Establish ground rules}

Establish clear ground rules and expectations at the start of the retreat. These include shared expectations for the group, such as setting quiet hours, assigning household responsibilities (e.g., cooking and cleaning), and setting guidelines for breaks. Ground rules should also set the tone of the retreat, creating a balanced foundation for respect, productivity, and enthusiasm.

At our retreat, we prioritised respectful and non-judgemental interactions between all participants. This attitude allowed flexibility in goal setting and accomplishments depending on each person's needs and stage of their PhD. Our baseline expectation of respect allowed us to set clear expectations for working versus quiet hours and to maintain a productive and peaceful environment. During our ground rule meeting, we also discussed the need for flexibility to ensure our retreat catered to the needs of all participants. For example, not all participants were native English speakers and may find full days of writing and discussion in English more tiring. As a result, we designated writing blocks as optional---if participants needed a longer break or felt they would benefit from a forest walk or gym session, they were free to do so without judgement. Continuing with our guideline of respect, we also made it clear in our opening meeting that while group discussions and feedback are useful, we would only tolerate constructive criticism. Together, clear expectations, a set schedule, and an overall attitude of respect set the tone for a successful week of writing. 

The document we used as the basis for our ground rules is in \nameref{S2_Appendix}.

\section*{Rule 6: Write}

Write. This is why you organise a retreat in the first place---to make progress on academic writing. Once a daily schedule and theme are set (Rule 3), it is time to execute. This is where individual goals (Rule 4) come into play. Knowing what you are working towards, and having a plan to get there, provides the structure needed to put words on the page and have productive writing blocks.

However, writing is not easy, and writers face many hurdles, such as a lack of motivation, uncertainty about what to do next, writer's block, imposter syndrome, and more ~\cite{boice1984, murray2013, wilson2019, grogan2021}. These obstacles can easily disrupt even the best-laid plans and most robust retreat structures. Therefore, have a backup plan. If it becomes challenging to write, consider taking a break or switching tasks; it is proven that even short breaks increase productivity ~\cite{carter2020, lyubykh2022}. Everyone has different strategies that work for them. For example, some people prefer to cite as they write, while others write everything out quickly and add citations during editing. Some authors, such as Ernest Hemingway, have even suggested to leave a sentence, paragraph, or idea unfinished at the end of a writing day so they have something easy to start with during the next writing block ~\cite{hemingway1935}. Assess what works best for your writing practice and reevaluate regularly to ensure that a strategy is still serving you ~\cite{peterson2018, grogan2021}.

\section*{Rule 7: Lead an academic discussion}

Schedule academic discussions around participants' work in a structured manner. A writing retreat gathers your academic peers in a collaborative environment with minimal external distractions. It presents a unique opportunity to garner advice and perspectives on written work (see also Rule 8) or any work-in-progress that participants may be developing. We suggest scheduling these discussions after writing blocks are complete for the day and spreading them out over the retreat. Access to a projector and screen in our second and third retreats was beneficial. However, previously, we have simply shared from our laptop screens or consulted notes and found it equally engaging due to the small group attending the retreats. Discussions were informal, and we aimed not to exceed 20 minutes per session. It is important to emphasise the casualness of such an exercise---students should not anxiously prepare a conference-level presentation at the expense of their writing time. These are simply opportunities to share, learn, and exchange insights. 

While we found academic discussions materialised naturally throughout the retreats, especially within groups of similarly focused researchers, we also allocated a small block of time each day (see Rule 3) specifically for students to share a talk or lead a discussion with the other attendees. The casual, non-judgemental environment allowed presenting students a chance to practice a seminar talk, gain insight on methodologies, analyses and results, or simply share `tips and tools of the trade.' Undertaking any large-writing project (e.g., a PhD thesis) is a daunting and varying experience for everyone. It is easy to become so engrossed in your work that you forget you are surrounded by people in a similar situation ~\cite{wilson2019}. A writing retreat provides the opportunity to share your own research or discuss scientific/academic topics amongst your peers. Importantly, these sessions offer a chance to get to know one another and connect as fellow PhD students ~\cite{omeara2017, bernery2022}.

\section*{Rule 8: Review and be reviewed}

Schedule time to exchange written material and review the work of other participants. An essential aspect of developing your writing is critical review both in terms of giving and receiving feedback as well as through editing your own work ~\cite{guilford2001, reynolds2011}. Constructive feedback from peers may offer valuable insights on gaps in clarity, logic, and structure. Even if you do not understand the scientific content of the work, focusing on structure, prose, and writing flow can be extremely helpful. This type of review can unblock a colleague who feels stuck. Likewise, feedback from a peer outside one's field may offer perspectives the author had not previously considered.

Providing feedback is just as valuable as receiving it. This practice encourages a collaborative approach to writing that builds confidence and resilience in both the reviewer and the reviewed. When giving feedback, use neutral language ~\cite{parsons2021} and focus on writing components, such as structure, clarity, and flow. If you are an expert in the content, detailed feedback can also be extremely valuable. 

Peer review is an integral part of academic life ~\cite{borja2024}, that needs to be practised. Furthermore, researchers should be mindful of the historical barriers peer review imposes on minority groups and non-English speakers ~\cite{smith2023, saleh2024}, and work towards breaking this pattern. A writing retreat is a friendly, safe space to talk about these barriers (Rule 7) and become more comfortable with the peer-review process in general while practising writing critical, neutral, and encouraging reviews.

\section*{Rule 9: Have fun}

Have fun and build connections with your peers. While on a writing retreat, recreational activities can enhance the overall experience. Furthermore, balancing downtime activities with writing blocks can lead to optimal productivity ~\cite{eardley2021, lyubykh2022}, improving the writing experience. A writing retreat allows you to disconnect from regular routines and responsibilities ~\cite{murray2013}, providing a unique opportunity to bond with your peers. Rural places often have scenic landscapes, hospitable communities and limited distractions. While this may restrict choices for activities, it can lead to higher-quality experiences where you can get to know local businesses and each other on a more personal level. Before embarking on a writing retreat, research what is unique about the location and take advantage of what may be on offer. 

While based at our university's field station, we had opportunities to explore the nearby Ulva Island (a protected area), participate in the locally famous bar trivia, and go on aurora hunts and kiwi spotting at night. In addition, we organised creative writing prompts each morning (see Rule 3; some of them hosted at the local cafe), which created fun moments to start the day. The casual presentations (Rule 7) about various topics were fun and engaging bonding activities as well.

\section*{Rule 10: Posterity---gather feedback and demonstrate value}

After the retreat, take time to reflect and gather feedback from participants. A group discussion about the retreat's highlights and challenges may help clarify thoughts before participants are asked to complete an anonymous survey. Shortly after returning, summarise the feedback into a concise report while details are still fresh. Be sure to include comments on the schedule and notes on things that went well or poorly. This is valuable for cementing ideas and details that would otherwise fade before another retreat can be planned. These insights will shape future retreats and ensure important lessons are not lost over time.

A report also creates accountability and continuity. Student organisers may change, but a shared archive of resources---such as schedules, ground rules, and facilitation tips---makes it easier to plan future events. Creating an institutional repository for retreat materials provides blueprints and templates to future organisers, setting them up for success.

In addition, publicly acknowledging the retreat in your writing and presentations demonstrates its academic value. Participants are encouraged to include the retreat in the acknowledgements of manuscripts and theses worked on during the retreat. Sharing this impact helps advocate for continued support, especially in institutions where publication output informs resource allocation. Maintaining a list of retreat-supported outputs can strengthen future proposals (see Rule 2) and make a compelling case for funding. Indeed, demonstrating the immense value of this experience may spur future support from leadership to fund retreats for students in other departments or institutions.

\section*{Conclusion}

Collectively, we experienced many of the benefits of writing retreats reported in the literature. These included feeling empowered as writers through supporting each other in a respectful and non-judgmental environment ~\cite{papen2018}, increasing our sense of belonging as PhD students ~\cite{omeara2017} and decreasing our sense of isolation ~\cite{eardley2021}. Most importantly, we achieved our writing goals at the retreats. Some of the references herein can help with the theoretical side of writing, but ultimately writing is practice and requires dedicated time ~\cite{peterson2018, grogan2021}. By facilitating the planning and execution of writing retreats through these Ten Simple Rules, we hope to foster scholarly productivity and wellbeing in student and early career cohorts. Happy writing!

\section*{Supporting information}

% Include only the SI item label in the paragraph heading. Use the \nameref{label} command to cite SI items in the text.
\paragraph*{S1 Appendix.}
\label{S1_Appendix}
{\bf Schedule Examples.} Daily schedules used during our writing retreats in 2023, 2024 and 2025.

\paragraph*{S2 Appendix.}
\label{S2_Appendix}
{\bf Ground Rules.} The base document for the Ground Rules discussed at the start of the retreats. Any additional points can be added to tailor the specifics of each group/location.

\section*{Acknowledgments}
This manuscript is a contribution conceptualised, led, and written by PhD students, who attended and organised writing retreats supported by the Department of Marine Science of the University of Otago, Aotearoa/New Zealand. We hope writing retreats stay as a trend in our Department, fostering collaboration and friendship among students and promoting quality science. Thanks also to Tracy Rogers and Claire Hudson from the Higher Education Development Centre of the University of Otago for the support while organising the first and second writing retreats, and the South Sea Hotel and the Rakiura/Stewart Island community for welcoming us. 

\section*{Author Contributions}

\paragraph*{Conceptualisation:} Nicholas W. Daudt (lead); Claudia Hird; Eleanor Kelly; Gretchen McCarthy; Ian S. Dixon-Anderson; Jackson Beagley; Jessica Moffitt; Joseph Curtis; Lindsay Wickman; Meghan Duffy; Saskia Foreman; Leah M. Crowe.

\paragraph*{Writing – original draft:} Nicholas W. Daudt; Claudia Hird; Eleanor Kelly; Gretchen McCarthy; Ian S. Dixon-Anderson; Jackson Beagley; Joseph Curtis; Meghan Duffy; Saskia Foreman; Leah M. Crowe.

\paragraph*{Writing – review & editing:} Nicholas W. Daudt (lead); Claudia Hird; Eleanor Kelly; Elli Leinikki; Gretchen McCarthy; Ian S. Dixon-Anderson; Jackson Beagley; Jessica Moffitt; Joseph Curtis; Lindsay Wickman; Meghan Duffy; Preston Maluafiti; Saskia Foreman; William Carome; Leah M. Crowe.

\paragraph*{Project administration:} Nicholas W. Daudt.

\nolinenumbers

% Either type in your references using
% \begin{thebibliography}{}
% \bibitem{}
% Text
% \end{thebibliography}
%
% or
%
% Compile your BiBTeX database using our plos2015.bst
% style file and paste the contents of your .bbl file
% here. See http://journals.plos.org/plosone/s/latex for 
% step-by-step instructions.

%\begin{thebibliography}{10}
%
%\bibitem{tydecks}
%Tydecks L, Bremerich V, Jentschke I, Likens GE, Tockner K.
%\newblock {{B}iological field stations: {A} global infrastructure for research, education, and public engagement}.
%\newblock BioScience 2016;66(2):164--171.
%
%\bibitem{hemingway}
%Hemingway E.
%\newblock {{M}onologue to the {M}aestro: {A} high seas letter}.
%\newblock Esquire, Oct; 1935.
%
%\bibitem{lyubykh}
%Lyubykh Z, Gulseren D, Premji Z, Wingate TG, Deng C, Bélanger LJ, Turner N.
%\newblock {{R}ole of work breaks in well-being and performance: {A} systematic review and future research agenda}.
%\newblock J. Occup. Health Psychol. 2022;27(5):470--487.
%
%\end{thebibliography}

\bibliography{references}


\end{document}

